\documentclass{article}
\usepackage{graphicx}
\usepackage{hyperref}
\usepackage{listings}
\usepackage{xcolor}

\title{HIPS Cutout Viewer User Guide}
\author{Yogesh Wadadekar}
\date{\today}

% Add copyright notice
\begin{document}

\maketitle

\begin{center}
\copyright{} 2024 Yogesh Wadadekar\\
\vspace{0.5em}
Project repository: \url{https://github.com/yogeshw/HIPSCutoutViewer}\\
\vspace{0.5em}
Licensed under the GPL 3 License
\end{center}

\section{Introduction}
The HIPS Cutout Viewer is a graphical application for retrieving and displaying astronomical image cutouts from various sky surveys using the Hierarchical Progressive Survey (HiPS) system.

\section{System Requirements}
\begin{itemize}
    \item Python 3.6 or later
    \item Required packages: PyQt6, astropy, astroquery, matplotlib
    \item Internet connection for accessing astronomical databases
\end{itemize}

\section{Getting Started}
\subsection{Launch the Application}
To start the application, run:
\begin{verbatim}
python hips_cutout_viewer.py
\end{verbatim}

\section{Main Interface}
The interface consists of several key areas:

\subsection{Input Section}
\begin{itemize}
    \item \textbf{Object Name:} Enter the name of an astronomical object (default: ``M 51'')
    \item \textbf{RA/Dec:} Direct input of coordinates in decimal degrees
    \item \textbf{Size:} Field of view in degrees (default: 0.1)
    \item \textbf{Resolve Name:} Button to obtain RA/Dec coordinates of chosen object using Simbad
\end{itemize}

\subsection{Survey Selection}
\begin{itemize}
    \item \textbf{Available Surveys:} Dropdown list of all available HiPS surveys
    \item \textbf{Selected Surveys:} List of surveys chosen for cutout retrieval
    \item \textbf{Default Surveys:} The following surveys are loaded by default:
        \begin{itemize}
            \item CDS/P/2MASS/color
            \item CDS/P/HST/EPO
            \item CDS/P/SDSS9/color
        \end{itemize}
    \item Use ``Add $\rightarrow$'' and ``$\leftarrow$ Remove'' buttons to manage survey selection
\end{itemize}

\subsection{Action Buttons}
\begin{itemize}
    \item \textbf{Get Cutouts:} Retrieve images for selected surveys
    \item \textbf{Save Collage:} Save the displayed images as a JPEG collage
    \item \textbf{Download FITS:} Save FITS format files for the cutouts
    \item \textbf{Reset:} Reset to default object (M 51) and default surveys while clearing other inputs
\end{itemize}

\section{Workflow}
\begin{enumerate}
    \item Enter an object name or coordinates
    \item If using a name, click ``Resolve Name'' to get coordinates
    \item Select desired surveys from the dropdown menu
    \item Adjust the cutout size if needed
    \item Click ``Get Cutouts'' to retrieve images
    \item Optionally save the collage or download FITS files
\end{enumerate}

\section{Features}
\subsection{Image Display}
\begin{itemize}
    \item Images are displayed in a grid layout
    \item Each image includes:
        \begin{itemize}
            \item North-East orientation arrows
            \item Scale bar (1 arcminute)
            \item Survey identification
        \end{itemize}
\end{itemize}

\subsection{FITS Downloads}
\begin{itemize}
    \item FITS files are saved in a 'fits' subdirectory
    \item Files are named according to their survey ID
    \item WCS information is preserved for scientific analysis
\end{itemize}

\section{Troubleshooting}
\begin{itemize}
    \item \textbf{Object Not Found:} Verify the object name in SIMBAD database
    \item \textbf{No Images:} Check if the selected surveys cover the requested region
    \item \textbf{Download Errors:} Verify internet connection and coordinates
\end{itemize}

\section{Tips}
\begin{itemize}
    \item The default object M 51 (Whirlpool Galaxy) provides a good starting point for exploring the interface
    \item Default surveys are preserved when using the Reset button
    \item Start with well-known objects to familiarize yourself with the interface
    \item Use smaller field of view values for detailed views
    \item Select complementary surveys for multi-wavelength analysis
\end{itemize}

\section{Technical Support}
For technical issues, feature requests, or contributions, please visit the project repository at:\\
\url{https://github.com/yogeshw/HIPSCutoutViewer}

You can submit issues, suggest improvements, or contribute to the development through pull requests. I don't have much bandwidth to work on this, so pull requests will receive the most attention!

\section{License}
This software is distributed under the GPL 3 License. The complete license text can be found in the LICENSE file in the project repository.

\end{document}
